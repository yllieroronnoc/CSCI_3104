% This is in AMSLaTeX.
\documentclass[10pt]{amsart}
\usepackage{amssymb}
% \renewcommand{\baselinestretch}{1.27}

\renewcommand{\S}{\subset}
\newcommand{\M}{\setminus}
\newcommand{\E}{\varnothing}
\renewcommand{\O}{\overline}
\newcommand{\I}{\infty}
\newcommand{\andeqn}{\qquad {\mbox{and}} \qquad}

\newcommand{\limi}[1]{\lim_{{#1} \to \infty}}

\theoremstyle{definition}
\newtheorem{thm}{Theorem}
\newtheorem{lem}[thm]{Lemma}
\newtheorem{prp}[thm]{Proposition}
\newtheorem{dfn}[thm]{Definition}
\newtheorem{cor}[thm]{Corollary}
\newtheorem{cnj}[thm]{Conjecture}
\newtheorem{cnv}[thm]{Convention}
\newtheorem{rmk}[thm]{Remark}
\newtheorem{ntn}[thm]{Notation}
\newtheorem{exa}[thm]{Example}
\newtheorem{pbm}[thm]{Problem}
\newtheorem{wrn}[thm]{Warning}
\newtheorem{qst}[thm]{Question}
\newtheorem{exr}[thm]{Exercise}

\newcommand{\af}{\alpha}
\newcommand{\bt}{\beta}
\newcommand{\gm}{\gamma}
\newcommand{\dt}{\delta}
\newcommand{\ep}{\varepsilon}
\newcommand{\zt}{\zeta}
\newcommand{\et}{\eta}
\newcommand{\ch}{\chi}
\newcommand{\io}{\iota}
\newcommand{\te}{\theta}
\newcommand{\ld}{\lambda}
\newcommand{\sm}{\sigma}
\newcommand{\kp}{\kappa}
\newcommand{\ph}{\varphi}
\newcommand{\ps}{\psi}
\newcommand{\rh}{\rho}
\newcommand{\om}{\omega}
\newcommand{\ta}{\tau}

\newcommand{\Gm}{\Gamma}
\newcommand{\Dt}{\Delta}
\newcommand{\Et}{\Eta}
\newcommand{\Th}{\Theta}
\newcommand{\Ld}{\Lambda}
\newcommand{\Sm}{\Sigma}
\newcommand{\Ph}{\Phi}
\newcommand{\Ps}{\Psi}
\newcommand{\Om}{\Omega}

\newcommand{\Q}{{\mathbb{Q}}}
\newcommand{\Z}{{\mathbb{Z}}}
\newcommand{\R}{{\mathbb{R}}}
\newcommand{\C}{{\mathbb{C}}}
% \newcommand{\N}{{\mathbb{N}}}
\newcommand{\N}{{\mathbb{Z}}_{> 0}}
\newcommand{\Nz}{{\mathbb{Z}}_{\geq 0}}

\pagenumbering{arabic}

\newcommand{\id}{{\mathrm{id}}}
\newcommand{\sint}{{\mathrm{int}}}
\newcommand{\diam}{{\mathrm{diam}}}
\newcommand{\dist}{{\mathrm{dist}}}

\newcommand{\A}{\qquad {\mbox{and}} \qquad}

\newcommand{\ct}{continuous}
\newcommand{\nbhd}{neighborhood}
\newcommand{\cpt}{compact}
\newcommand{\wolog}{without loss of generality}
\newcommand{\Wolog}{Without loss of generality}
\newcommand{\Tfae}{The following are equivalent}
\newcommand{\tfae}{the following are equivalent}
\newcommand{\ifo}{if and only if}
\newcommand{\ms}{metric space}
\newcommand{\cms}{compact metric space}
\newcommand{\cfn}{continuous function}
\newcommand{\hsa}{hereditary subalgebra}
\newcommand{\wrt}{with respect to}

\title{Math 413 [513] (Phillips) Solutions to Homework 7}
\date{12 November 2018}

\begin{document}

% \setcounter{section}{-1}

% \vspace*{-3ex}
\maketitle

Generally, a ``solution'' is something that would be acceptable
if turned in in the form presented here, although the solutions
given are often close to minimal in this respect.
A ``solution (sketch)'' is too sketchy to be considered a
complete solution if turned in;
varying amounts of detail would need to be filled in.

\vspace{3ex}

\noindent
{\textbf{Problem 3.11:}}
Let $(a_n)_{n \in \N}$ be a sequence in~$\R$
with $a_n > 0$ for all $n \in \N$.
For $n \in \N$
define $s_n = a_1 + a_2 + \cdots + a_n$.
Assume that $\sum_{n = 1}^{\infty} a_n$ diverges.

\vspace{1ex}

(a)
Prove that
\[
\sum_{n = 1}^{\infty} \frac{a_n}{1 + a_n}
\]
diverges.

\begin{proof}[Solution~1]
We consider two cases.
First,
assume $(a_n)_{n \in \N}$ is bounded.
Choose $M > 0$ such that for all $n \in \N$
we have $a_n < M$.
Then
\[
\frac{a_n}{1 + a_n} > \frac{a_n}{1 + M}.
\]
Now
\[
\sum_{n = 1}^{\infty} \frac{a_n}{1 + M}
\]
diverges,
so
\[
\sum_{n = 1}^{\infty} \frac{a_n}{1 + a_n}
\]
diverges by the Comparison Test.

Now suppose that $(a_n)_{n \in \N}$ is not bounded.
Then there is a subsequence $(a_{k (n)})_{n \in \N}$
such that $\limi{n} a_{k (n)} = \I$.
Then
\[
\limi{n} \frac{a_{k (n)}}{1 + a_{k (n)}} = 1.
\]
So
\[
\frac{a_n}{1 + a_n} \not\to 0,
\]
whence
\[
\sum_{n = 1}^{\infty} \frac{a_n}{1 + a_n}
\]
diverges.
\end{proof}

\begin{proof}[Solution~2]
We consider two cases.
First,
assume $\limi{n} a_n = 0$.
Choose $N \in \N$ such that for all $n \geq N$
we have $a_n < 1$.
For such~$n$,
we have
$1 + a_n < 2$,
so
\[
\frac{a_n}{1 + a_n} > \frac{a_n}{2}.
\]
Now
$\sum_{n = 1}^{\infty} a_n$ diverges,
so
\[
\sum_{n = N}^{\infty} \frac{a_n}{2}
\]
diverges (the summation now starts at~$N$,
but this doesn't change convergence).
Therefore
\[
\sum_{n = N}^{\infty} \frac{a_n}{1 + a_n}
\]
diverges by the Comparison Test.
Therefore
\[
\sum_{n = 1}^{\infty} \frac{a_n}{1 + a_n}
\]
diverges.

Before doing the other case,
we claim that if $x, y \in \R$ with $x > y > 0$,
then
\[
\frac{x}{1 + x} > \frac{y}{1 + y}.
\]
To prove the claim,
observe that
\[
\frac{x}{1 + x} - \frac{y}{1 + y}
 = \frac{x (1 + y)}{(1 + x)(1 + y)} - \frac{y (1 + x)}{(1 + x)(1 + y)}
 = \frac{x - y}{(1 + x)(1 + y)}
 > 0.
\]
The claim is proved.

Now suppose that $a_n \not\to 0$.
Then there is $\ep > 0$
such that for infinitely many $n \in \N$
we have $a_n > \ep$.
For these values of~$n$,
the claim implies
\[
\frac{a_n}{1 + a_n}
 > \frac{\ep}{1 + \ep}.
\]
Since $\ep / (1 + \ep) > 0$,
we conclude that
\[
\frac{a_n}{1 + a_n} \not\to 0,
\]
so
\[
\sum_{n = 1}^{\infty} \frac{a_n}{1 + a_n}
\]
diverges.
\end{proof}


(b)
Prove that for all $N, k \in \N$ we have
\[
\frac{a_{N + 1}}{s_{N + 1}} + \frac{a_{N + 2}}{s_{N + 2}}
    + \cdots + \frac{a_{N + k}}{s_{N + k}}
 \geq 1 - \frac{s_N}{s_{N + k}},
\]
and deduce that
\[
\sum_{n = 1}^{\infty} \frac{a_n}{s_n}
\]
diverges.

\begin{proof}[Solution]
Since the sequence $(s_n)_{n \in \N}$
is strictly increasing,
we have
%
\begin{align*}
\frac{a_{N + 1}}{s_{N + 1}} + \frac{a_{N + 2}}{s_{N + 2}}
    + \cdots + \frac{a_{N + k}}{s_{N + k}}
& > \frac{a_{N + 1}}{s_{N + k}} + \frac{a_{N + 2}}{s_{N + k}}
    + \cdots + \frac{a_{N + k}}{s_{N + k}}
\\
& = \frac{a_{N + 1} + a_{N + 2} + \cdots + a_{N + k}}{s_{N + k}}
\\
& = \frac{s_{N + k} - s_{N}}{s_{N + k}}
  = 1 - \frac{s_N}{s_{N + k}},
\end{align*}
%
as desired.

To prove divergence of
\[
\sum_{n = 1}^{\infty} \frac{a_n}{s_n},
\]
we show that the Cauchy criterion for convergence fails.
Take $\ep = \frac{1}{3}$.
Let $N \in \N$.
Since $\sum_{n = 1}^{\infty} a_n$ diverges,
we have $\limi{n} s_n = \I$,
so
\[
\limi{m} \left( 1 - \frac{s_N}{s_{m}} \right) = 0.
\]
Therefore there is $m > N$ such that
\[
1 - \frac{s_N}{s_{m}} < \frac{1}{3}.
\]
Now
\[
\sum_{k = N}^m \frac{a_k}{s_k}
 > 1 - \frac{s_N}{s_{m}}
 > \frac{2}{3}
 > \ep.
\]
This completes the solution.
\end{proof}

(c)
Prove that
\[
\frac{a_n}{s_n^2} \leq \frac{1}{s_{n - 1}} - \frac{1}{s_n}
\]
for $n \in \N$,
and deduce that
\[
\sum_{n = 1}^{\infty} \frac{a_n}{s_n^2}
\]
converges.

\begin{proof}[Solution]
For $n \in \N$ with $n \geq 2$, we have,
using $s_{n - 1} < s_n$ at the last step,
\[
\frac{1}{s_{n - 1}} - \frac{1}{s_n}
  = \frac{s_n - s_{n - 1}}{s_n s_{n - 1}}
  = \frac{a_n}{s_n s_{n - 1}}
  > \frac{a_n}{s_n^2}
  > 0.
\]
The first statement follows.

Now
%
\[
\sum_{n = 2}^{\infty} \left( \frac{1}{s_{n - 1}} - \frac{1}{s_n} \right)
 = \limi{N} \sum_{n = 2}^{N}
     \left( \frac{1}{s_{n - 1}} - \frac{1}{s_n} \right)
 = \limi{N} \left( \frac{1}{s_1} - \frac{1}{s_{N + 1}} \right).
\]
%
Since $\sum_{n = 1}^{\infty} a_n$ diverges,
we have $\limi{N} s_{N + 1} = \I$,
so
$\limi{N} 1 / s_{N + 1} = 0$.
Therefore
\[
\sum_{n = 2}^{\infty} \left( \frac{1}{s_{n - 1}} - \frac{1}{s_n} \right)
\]
converges.
Therefore
\[
\sum_{n = 1}^{\infty} \frac{a_n}{s_n^2}
\]
converges by the Comparison Test.
\end{proof}


(d)
What can be said about
\[
\sum_{n = 1}^{\infty} \frac{a_n}{1 + n a_n}
\andeqn
\sum_{n = 1}^{\infty} \frac{a_n}{1 + n^2 a_n}?
\]

\begin{proof}[Solution]
We first show that, depending on the choice of $(a_n)_{n \in \N}$,
the series
\[
\sum_{n = 1}^{\infty} \frac{a_n}{1 + n a_n}
\]
can either converge or diverge,

Define $a_n = \frac{1}{n}$ for $n \in \N$.
Then
\[
\sum_{n = 1}^{\infty} a_n
\]
diverges.
Also,
\[
\sum_{n = 1}^{\infty} \frac{a_n}{1 + n a_n}
  = \sum_{n = 1}^{\infty} \frac{1/n}{1 + n (1/n)}
  = \sum_{n = 1}^{\infty} \frac{1}{2 n},
\]
which diverges.

Now for $n \in \N$ define
\[
a_n = \begin{cases}
   \frac{1}{2^n}     & \hspace*{1em} {\mbox{$n$ is not a square}}
        \\
   \frac{1}{n^{1/3}} & \hspace*{1em} {\mbox{$n$ is a square}}
\end{cases}
\andeqn
c_n = \begin{cases}
   0                 & \hspace*{1em} {\mbox{$n$ is not a square}}
        \\
   \frac{1}{n^{1/3}} & \hspace*{1em} {\mbox{$n$ is a square}}.
\end{cases}
\]

We observe that
\[
\sum_{n = 1}^{\infty} c_n
 = \sum_{k = 1}^{\infty} c_{k^2}
 = \sum_{k = 1}^{\infty} \frac{1}{n^{2/3}}
 = \I.
\]
It now follows from the Comparison Test
that
$\sum_{n = 1}^{\infty} a_n$ diverges.

Next, for $n \in \N$ define
\[
x_n = \frac{1}{2^n}
\andeqn
y_n = \begin{cases}
   0           & \hspace*{1em} {\mbox{$n$ is not a square}}
        \\
   \frac{1}{n} & \hspace*{1em} {\mbox{$n$ is a square}}.
\end{cases}
\]
If $n$ is not a square,
then
\[
\frac{a_n}{1 + n a_n}
 = \frac{2^{-n}}{1 + 2^{-n} n}
 < 2^{-n}
 = x_n + y_n,
\]
while if $n$ is a square,
then
\[
\frac{a_n}{1 + n a_n}
 = \frac{n^{- 1/3}}{1 + n^{- 1/3} \cdot n}
 = \frac{n^{- 1/3}}{1 + n^{2/3}}
 <  \frac{n^{- 1/3}}{n^{2/3}}
 = \frac{1}{n}
 < x_n + y_n.
\]
Furthermore,
\[
\sum_{n = 1}^{\infty} (x_n + y_n)
 = \sum_{n = 1}^{\infty} x_n + \sum_{n = 1}^{\infty} y_n
 = \sum_{n = 1}^{\infty} x_n + \sum_{k = 1}^{\infty} y_{k^2}
 = \sum_{n = 1}^{\infty} \frac{1}{2^n}
    + \sum_{k = 1}^{\infty} \frac{1}{k^2}
 < \I.
\]
So
\[
\sum_{n = 1}^{\infty} \frac{a_n}{1 + n a_n}
\]
converges by the Comparison Test.

We now claim that
\[
\sum_{n = 1}^{\infty} \frac{a_n}{1 + n^2 a_n}
\]
always converges.
For $n \in \N$,
we have
$1 + n^2 a_n > n^2 a_n$,
so
\[
0 < \frac{a_n}{1 + n^2 a_n}
  < \frac{a_n}{n^2 a_n}
  = \frac{1}{n^2}.
\]
Since $\sum_{n = 1}^{\infty} \frac{1}{n^2}$ converges,
\[
\sum_{n = 1}^{\infty} \frac{a_n}{1 + n^2 a_n}
\]
converges by the Comparison Test.
\end{proof}


A full solution to the first part must give,
with proofs,
an example in which
\[
\sum_{n = 1}^{\infty} \frac{a_n}{1 + n a_n}
\]
converges
and an example in which this series diverges.




\vspace{2ex}

\noindent
{\textbf{Problem 3.13:}}
Suppose the complex series $\sum_{n = 0}^{\I} a_n$
and $\sum_{n = 0}^{\I} b_n$ converge absolutely.
For $n \in \Nz$ define
$c_n = \sum_{k = 0}^n a_k b_{n - k}$.
Prove that $\sum_{n = 0}^{\I} c_n$ converges absolutely.

\begin{proof}[Solution]
No solution has been written yet.
% ???
\end{proof}

\vspace{2ex}


\vspace{2ex}

\noindent
{\textbf{Problem 3.14:}}
Let $(s_n)_{n \in \Nz}$ be a sequence in~$\C$.
For $n \in \Nz$
define
\[
\sm_n = \frac{s_0 + s_1 + \cdots + s_n}{n + 1},
\]
the arithmetic mean of the terms with indices $0$ through~$n$.

\vspace{1ex}

(a)
Prove that if $\limi{n} s_n$ exists,
then $\limi{n} \sm_n = \limi{n} s_n$.

\begin{proof}[Solution]
For convenience,
define $s = \limi{n} s_n$.
Set $M = \sup_{n \in \Nz} | s_n |$.
Then also $| s | \leq M$.

Now let $\ep > 0$.
Choose $n_0 \in \N$ such that for all $n \geq n_0$,
we have $| s_n - s | < \frac{\ep}{2}$.
Choose $N \in \N$ such that
\[
N > \max \left( n_0, \, \frac{4 n_0 M}{\ep} \right).
\]
Let $n \geq N$.
Then
%
\begin{align*}
| \sm_n - s |
& = \left| \frac{1}{n + 1} \sum_{k = 0}^n (s_k - s) \right|
  \leq \frac{1}{n + 1} \sum_{k = 0}^n | s_k - s |
\\
& = \frac{1}{n + 1} \sum_{k = 0}^{n_0 - 1} | s_k - s |
     + \frac{1}{n + 1} \sum_{k = n_0}^n | s_k - s |
\\
& < \frac{1}{n + 1} \sum_{k = 0}^{n_0 - 1} 2 M
     + \frac{1}{n + 1} \sum_{k = n_0}^n \frac{\ep}{2}
\\
& = \frac{2 M n_0}{n + 1} + \frac{\ep (n - n_0 + 1)}{2 (n + 1)}
  < \frac{2 M n_0}{N} + \frac{\ep}{2}
  < \frac{\ep}{2} + \frac{\ep}{2}
  = \ep.
\end{align*}
%
This completes the solution.
\end{proof}

(b)
Construct an example in which $\limi{n} s_n$ does not exist
but $\limi{n} \sm_n = 0$.

\begin{proof}[Solution]
Define $s_n = (-1)^n$ for $n \in \Nz$.
Then
\[
\sm_n = \begin{cases}
   0               & \hspace*{1em} {\mbox{$n$ is even}}
        \\
   \frac{1}{n + 1} & \hspace*{1em} {\mbox{$n$ is odd}}.
\end{cases}
\]
Therefore $\limi{n} \sm_n = 0$.
However,
$(s_n)_{n \in \Nz}$ has two subsequential limits,
namely $1$ and $-1$,
so $\limi{n} s_n$ does not exist.
\end{proof}

\begin{proof}[Alternate solution]
The example given in the solution to Part~(c)
also works here.
\end{proof}

(c)
Is there an example in which $s_n > 0$ for all $n \in \Nz$,
$\limsup_{n \to \I} s_n = \I$,
and $\limi{n} \sm_n = 0$?

\begin{proof}[Solution]
Yes.

For convenience of writing,
it helps to write the example as the sum of two sequences.
(The notation would be simpler
if we could take some of the terms to be zero.)
Define $x_n = 2^{- n}$ for $n \in \Nz$.
For $k \in \Nz$ define $y_{2^{2 k}} = 2^k$,
and define $y_n = 0$ for all
$n \in \Nz \M \bigl\{ 1, 2^2, 2^4, 2^6, \ldots  \bigr\}$.
Then define $s_n = x_n + y_n$ for $n \in \Nz$.
We show that this sequence has the required properties.

It is obvious that $s_n > 0$ for all $n \in \Nz$.
It is also clear that
$\limi{k} s_{2^{2 k}} = \I$,
so $\limsup_{n \to \I} s_n = \I$.

It remains to prove that $\limi{n} \sm_n = 0$.
For $n \in \Nz$
define
\[
\xi_n = \frac{x_0 + x_1 + \cdots + x_n}{n + 1}
\andeqn
\et_n = \frac{y_0 + y_1 + \cdots + y_n}{n + 1}.
\]
Then $\sm_n = \xi_n + \et_n$.
We have $\limi{n} \xi_n = 0$ by Part~(a).
We calculate, for $k \in \Nz$
and $n \in \Z \cap [2^{2 k}, \, 2^{2 k + 2} - 1 ]$,
\[
0 \leq \et_{n}
  = \frac{1 + 2 + 2^2 + 2^3 + \cdots 2^{k}}{n + 1}
  = \frac{2^{k + 1} - 1}{n + 1}
  \leq \frac{2^{k + 1} - 1}{2^{2 k} + 1}
  < \frac{1}{2^{k - 1}}
  = \frac{4}{2^{k + 1}}
  < \frac{4}{\sqrt{n}}.
\]
Since $\limi{n} 4 / \sqrt{n} = 0$,
it follows that $\limi{n} \et_{n} = 0$.
This completes the solution.
\end{proof}


(d)
For $n \in \N$ define $a_n = s_n - s_{n - 1}$.
Prove that for $n \in \Nz$ we have
\[
s_n - \sm_n = \frac{1}{n + 1} \sum_{k = 1}^n k a_k.
\]
Using this fact,
prove that if $\limi{n} n a_n = 0$ and $\limi{n} \sm_n$ exists,
then $\limi{n} s_n$ exists.

\begin{proof}[Solution]
For the first part,
we take $a_0 = 0$ and prove by induction on~$n$ that
%
\begin{equation}\label{Eq_8Y10_NoDenom}
(n + 1) (\sm_n - s_n) = \sum_{k = 1}^n k a_k.
\end{equation}
%

This is true when $n = 0$,
since both sides are zero.

So assume (\ref{Eq_8Y10_NoDenom}) holds for~$n$;
we prove it for $n + 1$.
The definition of $(\sm_n)_{n \in \Nz}$
implies that
\[
(n + 2) \sm_{n + 1} = (n + 1) \sm_n + s_{n + 1}.
\]
Using this relation at the first step,
we get
%
\begin{align*}
& (n + 2) (s_{n + 1} - \sm_{n + 1}) - (n + 1) (s_n - \sm_n)
\\
& \hspace*{3em} {\mbox{}}
  = (n + 2) s_{n + 1} - (n + 1) \sm_n - s_{n + 1}
    - (n + 1) s_n + (n + 1) \sm_n
\\
& \hspace*{3em} {\mbox{}}
  = (n + 1) (s_{n + 1} - s_n)
  = (n + 1) a_{n + 1},
\end{align*}
%
so
\begin{align*}
(n + 2) (s_{n + 1} - \sm_{n + 1})
& = (n + 1) a_{n + 1} + (n + 1) (s_n - \sm_n)
\\
& = (n + 1) a_{n + 1} + \sum_{k = 1}^n k a_k
  = \sum_{k = 1}^{n + 1} k a_k.
\end{align*}
This completes the induction step,
and the proof of the first part.

For the second part,
it follows from Part~(a),
applied to the sequence $(n a_n)_{n \in \Nz}$,
that
\[
\limi{n} \frac{1}{n + 1} \sum_{k = 1}^n k a_k = 0.
\]
Applying the first part,
we conclude that
$\limi{n} (s_n - \sm_n) = 0$,
so
$\limi{n} s_n = \limi{n} \sm_n$.
\end{proof}

\begin{proof}[Alternate solution]
We have,
changing the order of summation at the second last step,
\begin{align*}
s_n - \sm_n
& = \frac{1}{n + 1} \sum_{k = 0}^{n - 1} (s_n - s_k)
\\
& = \frac{1}{n + 1} \sum_{k = 0}^{n - 1} \sum_{l = k + 1}^n a_l
  = \frac{1}{n + 1} \sum_{l = 0}^{n} \sum_{k = 0}^{l - 1} a_l
  = \frac{1}{n + 1} \sum_{l = 0}^{n} l a_l,
\end{align*}
as desired.
(One can check the change of order of summation by observing that,
in the first expression on the second line, $a_1$ occurs once
[with $k = 0$ and $l = 1$],
$a_2$ occurs twice [with $k = 0, 1$ and $l = 2$],
etc.)

For the second part,
we claim that
\[
\limi{n} \frac{1}{n + 1} \sum_{k = 1}^n k a_k = 0.
\]
To prove the claim,
set $M = \sup_{n \in \Nz} | n a_n |$.
Let $\ep > 0$.
Choose $n_0 \in \N$ such that for all $n \geq n_0$,
we have $| n a_n | < \frac{\ep}{2}$.
Choose $N \in \N$ such that
\[
N > \max \left( n_0, \, \frac{2 n_0 M}{\ep} \right).
\]
Let $n \geq N$.
Then
%
\begin{align*}
\left| \frac{1}{n + 1} \sum_{k = 1}^n k a_k \right|
& \leq \frac{1}{n + 1} \sum_{k = 1}^n | k a_k |
\\
& = \frac{1}{n + 1} \sum_{k = 0}^{n_0 - 1} | k a_k |
     + \frac{1}{n + 1} \sum_{k = n_0}^n | k a_k |
\\
& < \frac{1}{n + 1} \sum_{k = 0}^{n_0 - 1} M
     + \frac{1}{n + 1} \sum_{k = n_0}^n \frac{\ep}{2}
\\
& = \frac{M n_0}{n + 1} + \frac{\ep (n - n_0 + 1)}{2 (n + 1)}
  < \frac{M n_0}{N} + \frac{\ep}{2}
  < \frac{\ep}{2} + \frac{\ep}{2}
  = \ep.
\end{align*}
%
The claim is proved.
It now follows from the first part that
$\limi{n} s_n = \limi{n} \sm_n$.
\end{proof}


\vspace{2ex}

\noindent
{\textbf{Problem 3.19:}}
Define $x \colon \{ 0, 2 \}^{\N} \to \R$
by, for $a = (a_n)_{n \in \N}$,
\[
x (a) = \sum_{n = 1}^{\I} \frac{a_n}{3^n}.
\]
Prove that the range of this map is the Cantor set.

\begin{proof}[Solution]
We recall the inductive construction of the sets whose
intersection was defined to be the Cantor set~$K$.
Define
\[
\af_1^{(0)} = 0,
\qquad
\bt_1^{(0)} = 1,
\A
E_0 = \bigl[ \af_1^{(0)}, \, \bt_1^{(0)} \bigr],
\]
so that $E_0$ is the disjoint union of $2^0 = 1$
closed intervals, each of length $3^{- 0} = 1$.
Then, given
\[
E_n = \coprod_{k = 1}^{2^n} \bigl[ \af_k^{(n)}, \, \bt_k^{(n)} \bigr],
\]
with
\[
0 = \af_1^{(n)} < \bt_1^{(n)} < \af_2^{(n)} < \bt_2^{(n)}
  \cdots < \af_{2^n}^{(n)} < \bt_{2^n}^{(n)} = 1
\]
and
\[
\bt_k^{(n)} - \af_k^{(n)} = \frac{1}{3^n}
\]
for $k = 1, 2, \ldots, 2^n$,
define
\[
\af_{2 k - 1}^{(n + 1)} = \af_k^{(n)},
\qquad
\bt_{2 k - 1}^{(n + 1)} = \af_k^{(n)} + \frac{1}{3^{n + 1}},
\qquad
\af_{2 k}^{(n + 1)} = \af_k^{(n)} + \frac{2}{3^{n + 1}},
\]
and
\[
\bt_{2 k}^{(n + 1)}
 = \af_k^{(n)} + \frac{3}{3^{n + 1}}
 = \bt_k^{(n)}
\]
for $k = 1, 2, \ldots, 2^n$,
and define
\[
E_{n + 1}
 = \bigcup_{k = 1}^{2^{n + 1}}
     \bigl[ \af_k^{(n + 1)}, \, \bt_k^{(n + 1)} \bigr],
\]
which is a disjoint union.
Further recall that
\[
E_0 \supset E_1 \supset E_2 \supset \cdots
\A
K = \bigcap_{n = 0}^{\I} E_n.
\]

Now we consider the map~$x$.
First,
the series for $x (a)$ converges
and the sum is in $[0, 1]$ by comparison with
\[
\sum_{n = 1}^{\I} \frac{2n}{3^n} = 1.
\]

Next, let $a \in \{ 0, 2 \}^{\N}$.
For $n \in \Nz$ define
\[
x_n (a) = \sum_{k = 1}^{n} \frac{a_k}{3^k}
\A
l_n (a) = \sum_{k = 1}^{n} 2^{n - k} \left( \frac{a_k}{2} \right).
\]
We claim that
%
\begin{equation}\label{Eq_8Y11_Partx}
x_n (a) = \af_{1 + l_n (a)}^{(n)}
\end{equation}
%
for all $n \in \Nz$.
We prove the claim by induction on~$n$.
If $n = 0$ then both sides of~(\ref{Eq_8Y11_Partx}) are zero.
Suppose (\ref{Eq_8Y11_Partx}) is known for some $n \in \Nz$.
If $a_{n + 1} = 0$ then
$l_{n + 1} (a) = 2 l_n (a)$,
so
\[
x_{n + 1} (a)
  = x_n (a)
  = \af_{1 + l_n (a)}^{(n)}
  = \af_{2 (1 + l_n (a)) - 1}^{(n + 1)}
  = \af_{2 l_n (a) + 1}^{(n + 1)}
  = \af_{1 + l_{n + 1} (a)}^{(n + 1)}.
\]
If $a_{n + 1} = 2$ then
$l_{n + 1} (a) = 2 l_n (a) + 1$,
so
\[
x_{n + 1} (a)
  = x_n (a) + \frac{2}{3^{n + 1}}
  = \af_{1 + l_n (a)}^{(n)} + \frac{2}{3^{n + 1}}
  = \af_{2 (1 + l_n (a))}^{(n + 1)}
  = \af_{1 + l_{n + 1} (a)}^{(n + 1)}.
\]
Thus, (\ref{Eq_8Y11_Partx}) holds for $n + 1$.
The induction is complete,
and the claim is proved.

The claim implies that for all $n \in \Nz$ we have $x_n (a) \in E_n$.
Let $m \in \Nz$
Since $E_0 \supset E_1 \supset E_2 \supset \cdots$,
it follows that
$x_n (a) \in E_m$ whenever $n \geq m$.
Since $E_m$ is closed and $\limi{n} x_n (a) = x (a)$,
we get $x (a) \in E_m$.
This is true for all $m \in \Nz$,
so $x (a) \in \bigcap_{m = 0}^{\I} E_m = K$.

It remains to prove that for $y \in K$ there is $a \in \{ 0, 2 \}^{\N}$
such that $x (a) = y$.
For $n \in \Nz$,
we have $y \in E_n$,
so there is a unique $l (n) \in \{ 1, 2, \ldots, 2^n \}$
such that
\[
y \in \bigl[ \af_{l (n)}^{(n)}, \, \bt_{l (n)}^{(n)} \bigr].
\]
For $n \in \N$ define
\[
a_n = \begin{cases}
   0 & \hspace*{1em} {\mbox{$l (n)$ is odd}}
        \\
   2 & \hspace*{1em} {\mbox{$l (n)$ is even}}.
\end{cases}
\]

We claim that
%
\begin{equation}\label{Eq_8Y11_InInt}
\sum_{k = 1}^{n} \frac{a_k}{3^k} = \af_{l (n)}^{(n)}
\end{equation}
%
for all $n \in \Nz$.
We prove the claim by induction on~$n$.
For $n = 0$ we have $l (n) = 1$, so
\[
\sum_{k = 1}^{n} \frac{a_k}{3^k}
 = 0
% \in [0, 1]
% = \bigl[ \af_1^{(0)}, \, \bt_1^{(0)} \bigr].
 = \af_1^{(0)}.
\]
Suppose (\ref{Eq_8Y11_InInt}) is known for some $n \in \Nz$.
Since
\[
\bigl[ \af_{l (n)}^{(n)}, \, \bt_{l (n)}^{(n)} \bigr] \cap E_{n + 1}
 = \bigl[ \af_{2 l (n) - 1}^{(n + 1)},
            \, \bt_{2 l (n) - 1}^{(n + 1)} \bigr]
   \cup \bigl[ \af_{2 l (n)}^{(n + 1)},
            \, \bt_{2 l (n)}^{(n + 1)} \bigr],
\]
we have $l (n + 1) = 2 l (n) - 1$ or $l (n + 1) = 2 l (n)$.
In the first case, $l (n + 1)$ is odd,
so $a_{n + 1} = 0$,
and
\[
\sum_{k = 1}^{n + 1} \frac{a_k}{3^k}
 = \sum_{k = 1}^{n} \frac{a_k}{3^k}
 = \af_{l (n)}^{(n)}
 = \af_{2 l (n) - 1}^{(n)}
 = \af_{l (n + 1)}^{(n)}.
\]
In the second case, $l (n + 1)$ is even,
so $a_{n + 1} = 2$,
and
\[
\sum_{k = 1}^{n + 1} \frac{a_k}{3^k}
 = \sum_{k = 1}^{n} \frac{a_k}{3^k} + \frac{2}{3^{n + 1}}
 = \af_{l (n)}^{(n)} + \frac{2}{3^{n + 1}}
 = \af_{2 l (n)}^{(n)}
 = \af_{l (n + 1)}^{(n)}.
\]
The induction is complete, and the claim is proved.

Now let $n \in \Nz$.
We have
\[
0 \leq \sum_{k = n + 1}^{\I} \frac{a_k}{3^k}
  \leq \sum_{k = n + 1}^{\I} \frac{2}{3^k}
  = \frac{1}{3^{n}},
\]
so
\[
\sum_{k = 1}^{\I} \frac{a_k}{3^k}
 = \af_{l (n)}^{(n)} + \leq \sum_{k = n + 1}^{\I} \frac{a_k}{3^k}
 \in \bigl[ \af_{l (n)}^{(n)}, \, \bt_{l (n)}^{(n)} \bigr].
\]
Since $y$ is also in this interval,
it follows that
\[
\left| y - \sum_{k = 1}^{\I} \frac{a_k}{3^k} \right|
  \leq \bt_{l (n)}^{(n)} - \af_{l (n)}^{(n)}
  = \frac{1}{3^n}.
\]
Since this is true for all $n \in \Nz$,
we have
\[
\sum_{k = 1}^{\I} \frac{a_k}{3^k}
 = y,
\]
as was to be proved.
\end{proof}

\end{document}
